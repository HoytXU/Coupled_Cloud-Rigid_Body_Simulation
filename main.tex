\documentclass[11pt]{article}

% -------------------------------------------------------------
% Packages
% -------------------------------------------------------------
\usepackage{graphicx}
\usepackage{amsmath, amssymb, amsfonts}
\usepackage{hyperref}
\usepackage{geometry}
\usepackage{microtype}
\usepackage{setspace}

\geometry{margin=1in}
\setstretch{1.15}

% -------------------------------------------------------------
\begin{document}

\title{Cloud--Rigid Coupling Simulation}
\author{Hao Xu}
\date{November 2025}

\maketitle

% -------------------------------------------------------------
% Introduction (Original text with citations added)
% -------------------------------------------------------------
\section{Introduction}

Clouds are an important part of how we understand weather, atmosphere, and
even many natural scenes in movies. However, most computer graphics simulations 
treat clouds only as passive volumes that move with the wind and do not
respond when something passes through them 
\cite{Stam1999StableFluids, Fedkiw2001VisualSmoke, Dobashi2000Cloud, Harris2003GPUFluid}. 
This creates a gap between what actually happens in nature and what current graphics 
methods can show. To address this problem, we explore a simulation approach that allows 
rigid objects to physically interact with warm cloud volumes. The goal is to capture 
realistic effects, like air being pushed aside, turbulence forming behind the object,
and moisture changing because of heat, by using a fluid model that includes
condensation and latent heat 
\cite{Kessler1969WarmRain, Grabowski1999TwoMoment, Nishita1996MultipleScattering, Miyazaki2007PhysCloud}.

Realistic cloud-rigid body interaction is important not just for visual appeal.
In films, games, and educational tools, people often want scenes where an object
moves through a cloud, yet the physics behind this motion is almost always
simplified or ignored. This means artists, scientists, and students do not get an
accurate picture of how clouds would truly behave. Our project aims to narrow
this gap by simulating a rigid model, such as the plane, flying through a warm
cloud while exchanging momentum, heat, and moisture with the surrounding
air. By doing so, we hope to provide a method that benefits visual effects,
atmospheric education, and scientific storytelling 
\cite{Guendelman2005Coupling, Batty2007Variational, RobinsonMosher2011Aero}.

% -------------------------------------------------------------
% Bibliography
% -------------------------------------------------------------
\bibliographystyle{unsrt}
\bibliography{references}

\end{document}
